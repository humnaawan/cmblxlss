\documentclass{article} 
\usepackage[margin=0.8in]{geometry}
\usepackage{color}
\usepackage[dvipsnames]{xcolor}
\usepackage{hyperref}

\newcommand{\eq}[1]{\begin{equation}{#1}\end{equation}}
\newcommand{\clred}[1]{\textcolor{red}{#1}}
\newcommand{\done}[1]{\textcolor{ForestGreen}{#1}}


\begin{document}
\large{
\section*{Meeting Notes: May 30, 17}
% Based on daily log notes for May 30, 17
Overall goal: calculate cross-correlations between CMB lensing and LSS.
\begin{itemize}\item Want $C_\ell^{\kappa\kappa}$ which probes $\sigma_8^2$; $C_\ell^{\kappa g}$ which probes $b\sigma_8^2$; $C_\ell^{gg}$ which probes $b^2\sigma_8^2$.
	\begin{itemize}
	\item Since deal with galaxies in specific redshift bins, explicitly consider auto- and cross-spectra for the bins.
	\item Say $i$th $z$-bin has central redshift $z_i$. Adopt the convention of writing the cross spectrum as $C_\ell^{\kappa i}$ which probes $b(z_i) \sigma_8^2(z_i)$ and $C_\ell^{ii}$ which probes $b^2(z_i)\sigma_8^2(z_i)$. \clred{Should $\sigma_8$ have the $z$-dependence? If yes, if/how does $z$-bin dependence come in through $C_\ell^{\kappa\kappa}$?}
          \clred{MM: $z$-dependence of $\sigma_8$ comes in through the growth factor. $P(k,z)\approx D^2(z)P(k,z_0)$ and $C_\ell^{\kappa\kappa}$ in turn depends on it through $C_\ell^{\kappa\kappa}=\int dz W^2(z) P(k=\ell/\chi,z)$}
	\end{itemize}
\item Expect coupled systematics due to dust uncertainties to be the biggest source of problems. Need to look at cross-correlations between CMB lensing and dust uncertainties.
\item Will need to think about two kinds of dust contributions: 1) MW dust extinguishes background galaxies; its microwave emission affects lensing, and 2) CIB affects lensing and hence correlates with galaxy distribution.
\end{itemize}

% Plan
% kappa
\subsection*{Convergence Field}
Know
\eq{\kappa(x)= \int_0^\infty dz W^c(z)\delta(x,z) \label{eq: kappa}}
where $W^c(z)$ is the CMB window function, $\delta(x,z)$ is the matter density field, and $x$ is the position on the sky.

Now, since we have LSS for different redshift bins and hence have $\delta(x, z)$ in Equation~\ref{eq: kappa}, we can break $\kappa(x)$ as contributions from different redshift bins:
\eq{\kappa(x)= \int_0^{z_1} dz W^c(z)\delta(x,z) + \int_{z_1}^{z_2} dz W^c(z)\delta(x,z) + ... + \int_{z_n}^\infty dz W^c(z)\delta(x,z) \label{eq: kappa break}}

We can choose redshift bins such that the CMB window function $W^c(z)$ can be approximated as a top-hat in the given bin (with central redshift $z_i$). Then we have
\eq{\kappa(x)= \sum_{i=1}^n W(z_i)\delta(x,z_i) +  \int_{z_n}^\infty dz W^c(z)\delta(x,z) \label{eq: kappa break}}
where the last integral needs to be evaluated in full since neither $W^c(z)$ nor $\delta(x,z) $ is a constant for any broad $z$-bin.

% cross correlation
\subsection*{Cross-Correlations}
In Limber approximation, the cross-correlation can be written in closed form as
\eq{C_\ell^{\kappa g}= \int_0^\infty dz W^g(z)W^c(z)b(z)P^{true}_\ell(k, z) \label{eq: C-cross}}
where $W^g(z)$ is the LSS window function, $W^c(z)$ is the CMB window function and $P^{true}_\ell(k, z)$ is the true matter power spectrum as a function of redshift.

% OS induced power
\subsection*{Observed Power Spectrum}
We can use Equation~\ref{eq: C-cross} to incorporate the effects of artifacts induced in the observed matter power spectrum. For instance, we know From Awan+ 2016 and LSST Observing Strategy White Paper that
\eq{P^{obs}_\ell(k, z)= W_\ell^2 P^{true}_\ell(k, z) + P^{OS}_\ell(k, z) \label{eq: Pobs}}
where $W_\ell$ is the survey window function.

% Plan
\subsection*{Overall Plan}
\begin{itemize}
\item Use \href{https://github.com/msyriac/orphics/blob/1153f3e6b634b780cb7676444b7312df4b2ea7f6/orphics/theory/cosmology.py}{\texttt{orphics}} to find the auto- and cross-correlations using analogs of Equation~\ref{eq: C-cross}, i.e., $C_\ell^{\kappa\kappa}, C_\ell^{\kappa i}, C_\ell^{ii}$. For $W^g$, start with assuming its a top-hat in each $z$-bin.
\item Get realizations of $\kappa$ and density maps using synfast: input $C_\ell^{\kappa\kappa}, C_\ell^{\kappa i}, C_\ell^{ii}$ as the TE fields.
\item Create a lensed CMB map using the realized $\kappa$ map.
\item Add dust artifacts to the lensed CMB map and LSS maps. Adding dust to lensing CMB maps makes sure dust contamination enters through the bispectrum $\langle TTg \rangle$ as expected.
\item Add OS artifacts to the realized map of LSS. These artifacts are calculate using LSST Metric Analysis Framework pipeline.
\item Cross-correlate the with-artifacts maps to see the amount of spurious power.
\end{itemize}


% To-Do
\subsection*{To-Do}
\begin{itemize}
\item Humna: Look at Baxter+ (SPT/DES), Miyatake+ (Planck/CMASS), Schaan et al.
\item \done{Humna: Set up CAMB.}
\item \done{Humna: Set up \texttt{orphics} to get $W^c(z)$.}
\item Humna: update the current work to use \texttt{orphics} for top hat LSS window function. Compare results (current) without the pipeline,
\item Need to figure out dust systematics for both CMB optical frequencies. Who to talk to? David Alonso; Alex Van Engelene, Reza Ansari?
\item Humna: provide Mat with $\kappa$ and dust maps.
\item Mat: Create the lensed CMB map and return reconstructed $\kappa$ map.
\item Humna: Run LSST OS artifacts pipeline at Nside= 1024. No need to get higher-$z$ spectra from Hu as wont be using his data as true LSS but CAMB's. Nelson's mock catalogs have high-$z$ spectra so need to incorporate them to calculate the artifacts for all relevant redshift bins.
\item All: Do a literature review to see what has been done in the field so far.
\end{itemize}









}
\end{document}
